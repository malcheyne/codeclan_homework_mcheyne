% Options for packages loaded elsewhere
\PassOptionsToPackage{unicode}{hyperref}
\PassOptionsToPackage{hyphens}{url}
%
\documentclass[
]{article}
\title{NHS Borders Cancer Report}
\author{}
\date{\vspace{-2.5em}}

\usepackage{amsmath,amssymb}
\usepackage{lmodern}
\usepackage{iftex}
\ifPDFTeX
  \usepackage[T1]{fontenc}
  \usepackage[utf8]{inputenc}
  \usepackage{textcomp} % provide euro and other symbols
\else % if luatex or xetex
  \usepackage{unicode-math}
  \defaultfontfeatures{Scale=MatchLowercase}
  \defaultfontfeatures[\rmfamily]{Ligatures=TeX,Scale=1}
\fi
% Use upquote if available, for straight quotes in verbatim environments
\IfFileExists{upquote.sty}{\usepackage{upquote}}{}
\IfFileExists{microtype.sty}{% use microtype if available
  \usepackage[]{microtype}
  \UseMicrotypeSet[protrusion]{basicmath} % disable protrusion for tt fonts
}{}
\makeatletter
\@ifundefined{KOMAClassName}{% if non-KOMA class
  \IfFileExists{parskip.sty}{%
    \usepackage{parskip}
  }{% else
    \setlength{\parindent}{0pt}
    \setlength{\parskip}{6pt plus 2pt minus 1pt}}
}{% if KOMA class
  \KOMAoptions{parskip=half}}
\makeatother
\usepackage{xcolor}
\IfFileExists{xurl.sty}{\usepackage{xurl}}{} % add URL line breaks if available
\IfFileExists{bookmark.sty}{\usepackage{bookmark}}{\usepackage{hyperref}}
\hypersetup{
  pdftitle={NHS Borders Cancer Report},
  hidelinks,
  pdfcreator={LaTeX via pandoc}}
\urlstyle{same} % disable monospaced font for URLs
\usepackage[margin=1in]{geometry}
\usepackage{graphicx}
\makeatletter
\def\maxwidth{\ifdim\Gin@nat@width>\linewidth\linewidth\else\Gin@nat@width\fi}
\def\maxheight{\ifdim\Gin@nat@height>\textheight\textheight\else\Gin@nat@height\fi}
\makeatother
% Scale images if necessary, so that they will not overflow the page
% margins by default, and it is still possible to overwrite the defaults
% using explicit options in \includegraphics[width, height, ...]{}
\setkeys{Gin}{width=\maxwidth,height=\maxheight,keepaspectratio}
% Set default figure placement to htbp
\makeatletter
\def\fps@figure{htbp}
\makeatother
\setlength{\emergencystretch}{3em} % prevent overfull lines
\providecommand{\tightlist}{%
  \setlength{\itemsep}{0pt}\setlength{\parskip}{0pt}}
\setcounter{secnumdepth}{-\maxdimen} % remove section numbering
\ifLuaTeX
  \usepackage{selnolig}  % disable illegal ligatures
\fi

\begin{document}
\maketitle

\hypertarget{intordution}{%
\section{Intordution}\label{intordution}}

Report written \n by \n Malcolm Cheyne

\hypertarget{brakedown-of-cancers-in-nhs-borders}{%
\section{Brakedown of Cancers in NHS
Borders}\label{brakedown-of-cancers-in-nhs-borders}}

There are 51 types of cancer that are recorded across Borders that
affect both men and women. Out of all these types of cancer the majority
have less than 50 cases a year in NHS Borders.

Taking the top 7 types of cancer that are have more than 50 cases a year
in NHS Borders.

\begin{itemize}
\tightlist
\item
  Non-melanoma skin cancer\\
\item
  Basal cell carcinoma of the skin\\
\item
  Breast\\
\item
  Trachea, bronchus and lung\\
\item
  Colorectal cancer\\
\item
  Squamous cell carcinoma of the skin\\
\item
  Colon
\end{itemize}

\includegraphics{NHS-Borders-Cancer-Report_files/figure-latex/unnamed-chunk-3-1.pdf}

We can see a steady rise's for two skin cancers incidences over the 24
year period. Breast cancer comes in peaks that mite have come from
increased publicity encouraging people to check or be tested in the lead
up to the peaks. They can be split by proactive and reactive approaches.

\hypertarget{proactive}{%
\subsection{Proactive}\label{proactive}}

\begin{itemize}
\item
  We can see that three of them are based on skin cancers that have like
  developed from exposed to ultraviolet (UV) radiation from the sun.
  This can be combated with more publicity encouraging on when and how
  to using sunscreen, while in the local area or on holiday abroad.
\item
  Trachea, bronchus and lung cancers, came mainly from direct or passive
  smoking. This can be combated with more publicity on the dangers of
  direct or passive smoking to people and their family. Can look at
  putting more support for people trying to quit.
\end{itemize}

\hypertarget{reactive}{%
\subsection{Reactive}\label{reactive}}

\begin{itemize}
\item
  2 are types Bowel cancers, while the cause of these is not known,
  certain risk factors are strongly linked to the disease, including
  diet, tobacco smoking and heavy alcohol use.
\item
  Breast cancer is caused when the DNA in breast cells mutate or change,
  disabling specific functions that control cell growth and division.
\end{itemize}

\hypertarget{cancer-by-age-and-sex-demographics}{%
\section{Cancer by Age and Sex
Demographic's}\label{cancer-by-age-and-sex-demographics}}

\includegraphics{NHS-Borders-Cancer-Report_files/figure-latex/unnamed-chunk-4-1.pdf}

Here we can see the brake down by age and gender showing the older the
person is the greater risk they have. While women start by being at
higher risk at a younger age older men make up the highest risk of
cancer. This adds these ages to the demographic of targets to focus any
publicity for proactive and reactive approaches. Both genders peak in
the 70-74 range and men have the over take women as having the highest
risk in the 55-59 range.

\includegraphics{NHS-Borders-Cancer-Report_files/figure-latex/unnamed-chunk-5-1.pdf}

\includegraphics{NHS-Borders-Cancer-Report_files/figure-latex/unnamed-chunk-6-1.pdf}

This is a brake down of Cancer type by age reinforces the demographic of
targets to focus any publicity for proactive and reactive approaches.

\begin{itemize}
\item
  Skin and lung cancers both men and women being encouraged to take
  proactive while young.
\item
  Breast and prostate cancers for women and men respectively should be
  checked regularly at these ages.
\end{itemize}

\includegraphics{NHS-Borders-Cancer-Report_files/figure-latex/unnamed-chunk-7-1.pdf}
\includegraphics{NHS-Borders-Cancer-Report_files/figure-latex/unnamed-chunk-7-2.pdf}

While almost all these cancers are shared equally between the two
genders, two cancers breast and prostate have a traditional gender bias
on which is effected. As-well two of the skin cancers have a higher risk
in men since around 2005 on wards. This can be a demographic to target
to focus the publicity encouraging talked about above or showing the
risk

\hypertarget{health-boards-comparason}{%
\section{Health Boards comparason}\label{health-boards-comparason}}

\includegraphics{NHS-Borders-Cancer-Report_files/figure-latex/unnamed-chunk-8-1.pdf}

In comparison to other local Health Boards, NHS Borders has a small
percentage of the cancer incidences across Scotland

\hypertarget{allover-trend-v-crude-rate}{%
\section{Allover trend v Crude rate}\label{allover-trend-v-crude-rate}}

\includegraphics{NHS-Borders-Cancer-Report_files/figure-latex/unnamed-chunk-9-1.pdf}

Focusing in on NHS Borders we can see a steady rise in the number of
cancer incidences over the 24 year period. The crude rate that is
measured by per 100,000 has lagged behind and not followed this increase
since around 2004.

The shaded area shows the confidence interval defined as a specified
probability that the value of a parameter lies within it. The shaded
area shows 5-95\% of the Crude Rate.

From about 2004 onwords the number of cancer incidences exceeds the
confidence interval. This seams to show a significant difference from
the expected number of cancer incidences for this size of population.
This abnormal rise in the cancer incidences away from the expected
number could be from the increase on the two skin cancers seen in the
first graph.

\hypertarget{references}{%
\section{References}\label{references}}

The data used were all sourced from
\url{https://www.opendata.nhs.scot/}. The links below are all the
specific links to each dataset used:

\url{https://www.opendata.nhs.scot/dataset/c2c59eb1-3aff-48d2-9e9c-60ca8605431d/resource/3aef16b7-8af6-4ce0-a90b-8a29d6870014/download/opendata_inc9519_hb.csv}
- Incidence by Health Board

\url{https://www.opendata.nhs.scot/dataset/c2c59eb1-3aff-48d2-9e9c-60ca8605431d/resource/e8d33b2b-1fb2-4d59-ad21-20fa2f76d9d5/download/opendata_inc1519comb_hb.csv}
- 5 Year Summary of Incidence by Health Board

\url{https://www.nhsscotlandci.scot.nhs.uk/wp-content/uploads/Shared/pdf/identity_guidelines.pdf}
- Colour Palette

\end{document}
